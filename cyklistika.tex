\documentclass[11pt]{article}
\usepackage[czech]{babel}
\usepackage[utf8]{inputenc}
\usepackage[T1]{fontenc}
\usepackage{a4wide}

\begin{document}
\begin{titlepage}
    \begin{center}
        \vspace*{1cm}

        \Huge
        \textbf{{Cyklistika}}

        \vspace{0.5cm}
        ... a vše, co jste potřebovali vědět.

        \vspace{1.5cm}

        \large
        \textbf{Robert Blažek}
        
        \vfill

        Seminární práce pro dodatečnou klasifikaci z tělesné výchovy

        \vspace{0.8cm}
        GJR Chrudim\\
        Nasavrky, 2020
    \end{center}
\end{titlepage}

\vspace*{\fill}
\noindent © Robert Blažek, 2020
\newline \newline
Prohlašuji, že jsem tuto práci vypracoval samostatně s~použitím uvedených pramenů.
\newline \newline \newline \newline 
\begin{tabular}{@{}p{0in}p{4in}@{}}
    & \hrulefill \\
    & Robert Blažek \\
    & Nasavrky, 2020\\
    \end{tabular}
\newpage
\tableofcontents
\newpage

\section{Úvod}
V této práci bych rád pokryl historii a druhy jízdních kol, druhy závodní cyklistiky a~problematiku správného výběru a péče o jízdní kolo. Dále se chci zaměřit na správné zvládnutí základních dovedností, jako jsou např. pohled a posed, průjezd zatáčkou, práce s těžištěm a~řazení. Práce se soustředí hlavně na nejpopulárnější horská kola.
\subsection[Stručná historie]{Stručná historie cyklistiky}
Jízdní kolo bylo vynalezeno na počátku 19. století. Jednalo se o tzv. drezínu podobnou dnešnímu odrážedlu s dřevěnými koly, které jízdu dělaly obzvlášť nepohodlnou. V~druhé polovině 19. stolení byly přidány pedály pevně spojené s~předním kolem a~objevily se první pokusy o~změkčení jízdy s využitím gumových plášťů a odpruženého sedla.
\paragraph*{} Za učelem zvýšení efektivity šlapání a rychlosti postupně docházelo ke zvětšování předního kola, čímž vzniklo tzv. vysoké kolo. Jeho nesporná nevýhoda však spočívala ve vysokém těžišti (o~něm ještě uslyšíme), a~s~ním spojené nestabilitě. 
\paragraph*{} Teprve na začátku 20. století se začíná objevovat kolo současného typu s~řetězem, pneumatikami a lankovými brzdami. Dalším mezníkem byl až vynález horského kola na přelomu 70.~a~80.~let.
\subsection[Druhy kol]{Druhy jízdních kol}
\textit{Moderní jízdní kola se nejčastěji dělí podle funkce. Pro zjednodušení uvádím obceně nejrozšířenější standardní druhy používané v~cyklistickém sportu.}
\paragraph*{Horské kolo}je v~současnosti asi nejpopulárnějším druhem jízdního kola. Vyznačuje se zejména širokými plášti a~velmi odolným rámem a komponenty určenými pro jízdu v terénu. Obvykle má odpruženou vidlici, dostatek převodů pro příkrá stoupání a spolehlivé diskové brzdy pro prudké sjezdy.
\paragraph*{Silniční kolo}je standardem ve světě rychlostní cyklistiky. Je optimalizované pro minimální odpor a maximální efektivitu na hladkém povrchu. Navzdory vysokým rychlostem často nemají závodní silniční kola diskové brzdy, ale horší a~méně spolehlivé špalíkové brzdy, které způsobují hromadné kolize v silničních závodech jako např. \textit{Le Tour de France}.
\paragraph*{Trekové/krosové kolo}je kompromisem mezi horským a silničním kolem. Má menší odpor na silnici a odolnější konstrukci do lehčího terénu.
\paragraph*{Trialové kolo}neboli BMX je speciální kategorie kol určená k jízdě na překážkách a bikrosových tratích. Vyznačuje se malými koly, jednoduchou konstrukcí a~stupačkami umožňujícími triky a jízdu po předním/zadním kole.
\newpage
\section[Výběr kola]{Jak vybrat kolo}
\subsection{Průměr kol}
Průměr samotného kola se udává v palcích ("). Záleží na něm hlavně u~dětských kol - u~kol pro dospělé se velikosti pohybují od 26"~do 29"~a záleží hlavně na schopnostech jezdce. 
\newline Je nevhodné, aby malé děti jezdily na kole, které je příliš velké nebo malé. V závodním prostředí je však běžné dětem pořizovat větší kola pro jejich lepší jízdní vlastnosti. 
\begin{table}[h]
    \centering
    \begin{tabular}{|c | c | c |} 
        \hline
        Průměr kola (")& Výška postavy (cm) & Orientační věk \\ 
        \hline\hline
        20 & 110 - 130 & 6 - 8 \\ 
        \hline
        24 & 125 - 150 & 8 - 10 \\
        \hline
        26 & nad 140 & nad 10 let \\
        \hline
    \end{tabular}
\caption{\label{tab:detskakola}Doporučené velikosti dětských kol}
\end{table}

\subsection{Výbava kola}

\newpage
\section{Technika jízdy}

\subsection{Pohled}

\end{document}