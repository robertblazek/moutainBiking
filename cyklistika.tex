\documentclass[11pt]{article}
\usepackage[czech]{babel}
\usepackage[utf8]{inputenc}
\usepackage[T1]{fontenc}
\usepackage{a4wide}
\usepackage{graphicx}
\usepackage{wrapfig}
\usepackage{cite}
\usepackage{url}
\graphicspath{ {./img/} }

\begin{document}
\begin{titlepage}
    \begin{center}
        \vspace*{1cm}

        \Huge
        \textbf{{Cyklistika}}

        \vspace{0.5cm}
        ... a vše, co jste potřebovali vědět.

        \vspace{1.5cm}

        \large
        \textbf{Robert Blažek}
        
        \vfill

        Seminární práce pro dodatečnou klasifikaci z tělesné výchovy

        \vspace{0.8cm}
        GJR Chrudim\\
        Nasavrky, 2020
    \end{center}
\end{titlepage}

\vspace*{\fill}
\noindent © Robert Blažek, 2020
\newline \newline
Prohlašuji, že jsem tuto práci vypracoval samostatně s~použitím uvedených pramenů. \newline Všechny obrázky, není-li uvedeno jinak, jsou mým dílem.
\newline \newline \newline \newline 
\begin{tabular}{@{}p{0in}p{4in}@{}}
    & \hrulefill \\
    & Robert Blažek \\
    & Nasavrky, 2020\\
    \end{tabular}
\newpage
\tableofcontents
\newpage

\section{Úvod}
V této práci bych rád pokryl historii a druhy jízdních kol, druhy závodní cyklistiky a~problematiku správného výběru a péče o jízdní kolo. Dále se chci zaměřit na správné zvládnutí základních dovedností, jako jsou např. pohled, průjezd zatáčkou a práce s těžištěm. Práce se soustředí hlavně na nejpopulárnější horská kola.
\subsection[Stručná historie]{Stručná historie cyklistiky}
Jízdní kolo bylo vynalezeno na počátku 19. století. Jednalo se o tzv. drezínu podobnou dnešnímu odrážedlu s dřevěnými koly, které jízdu dělaly obzvlášť nepohodlnou. V~druhé polovině 19. stolení byly přidány pedály pevně spojené s~předním kolem a~objevily se první pokusy o~změkčení jízdy s využitím gumových plášťů a odpruženého sedla.\cite{kolowiki}
\paragraph*{} Za učelem zvýšení efektivity šlapání a rychlosti postupně docházelo ke zvětšování předního kola, čímž vzniklo tzv. vysoké kolo. Jeho nesporná nevýhoda však spočívala ve vysokém těžišti (o~něm ještě uslyšíme), a~s~ním spojené nestabilitě. \cite{kolowiki}
\paragraph*{} Teprve na začátku 20. století se začíná objevovat kolo současného typu s~řetězem, pneumatikami a lankovými brzdami. Dalším mezníkem byl až vynález horského kola na přelomu 70.~a~80.~let. \cite{kolowiki}
\subsection[Druhy kol]{Druhy jízdních kol}
\textit{Moderní jízdní kola se nejčastěji dělí podle funkce. Pro zjednodušení uvádím obceně nejrozšířenější standardní druhy používané v~cyklistickém sportu.}
\paragraph*{Horské kolo}je v~současnosti asi nejpopulárnějším druhem jízdního kola. Vyznačuje se zejména širokými plášti a~velmi odolným rámem a komponenty určenými pro jízdu v terénu. Obvykle má odpruženou vidlici, dostatek převodů pro příkrá stoupání a spolehlivé diskové brzdy pro prudké sjezdy.
\paragraph*{Silniční kolo}je standardem ve světě rychlostní cyklistiky. Je optimalizované pro minimální odpor a maximální efektivitu na hladkém povrchu. Navzdory vysokým rychlostem často nemají závodní silniční kola diskové brzdy, ale horší a~méně spolehlivé špalíkové brzdy, které způsobují hromadné kolize v silničních závodech jako např. \textit{Le Tour de France}.
\paragraph*{Trekové/krosové kolo}je kompromisem mezi horským a silničním kolem. Má menší odpor na silnici a odolnější konstrukci do lehčího terénu.
\paragraph*{Trialové kolo}neboli BMX je speciální kategorie kol určená k jízdě na překážkách a bikrosových tratích. Vyznačuje se malými koly, jednoduchou konstrukcí a~stupačkami umožňujícími triky a jízdu po předním/zadním kole.
\newpage
\section[Výběr kola]{Jak vybrat horské kolo}
\begin{figure}[h]
    \centering
    \includegraphics[width=0.8\textwidth]{mtb}
    \caption{Horské kolo  \tiny{foto: Specialized.com}}
    \label{fig:mtb}
\end{figure}
\subsection{Průměr kol}
Průměr samotného kola se udává v palcích ("). Záleží na něm hlavně u~dětských kol - u~kol pro dospělé se velikosti pohybují od 26"~do 29"~a záleží hlavně na schopnostech jezdce. 
\newline Je nevhodné, aby malé děti jezdily na kole, které je příliš velké nebo malé. V závodním prostředí je však běžné dětem pořizovat větší kola pro jejich lepší jízdní vlastnosti. 
\begin{table}[h]
\caption{\label{tab:detskakola}Doporučený průměr dětských kol \cite{velikostiKol}}
    \centering
    \begin{tabular}{|c | c | c |} 
        \hline
        Průměr kola (")& Výška postavy (cm) & Orientační věk \\ 
        \hline\hline
        20 & 110 - 130 & 6 - 8 \\ 
        \hline
        24 & 125 - 150 & 8 - 10 \\
        \hline
        26 & nad 140 & nad 10 let \\
        \hline
    \end{tabular}

\end{table}
\subsection{Velikost rámu}
Velikost rámu se rovněž udává v palcích a určuje se podle následující tabulky.
\begin{table}[h]
\caption{\label{tab:velikostRamu}Doporučené velikosti rámu \cite{velikostiKol}}
    \centering
    \begin{tabular}{|c | c | c |} 
        \hline
        Výška postavy (cm) & Velikost rámu (") & Velikost rámu \\ 
        \hline\hline
        140 - 160 & 14 - 15 & XS \\ 
        \hline
        160 - 172 & 15,5 - 16,5 & S \\
        \hline
        168 - 180 & 17 - 18 & M \\
        \hline
        177 - 187 & 18,5 - 19,5 & L \\
        \hline
        184 - 197 & 20 - 21,5 & XL \\
        \hline
        195 - ?~~ & 22 - 23 & XXL \\
        \hline
    \end{tabular}

\end{table}

\newpage\subsection[Výbava jízdního kola]{Výbava kola}
\begin{figure}[h]
    \centering
    \includegraphics[width=0.95\textwidth]{kolo}
    \caption{Součásti jízdního kola  \tiny{foto: wikipedie, autor: Niabot – Image:Bicycle\_diagram-en.svg}}
    \label{fig:kolo}
\end{figure}
\subsubsection*{Materiál}
Rámy jízdních kol se nejčastěji vyrábí ze slitiny hliníku. Ten se ve slitinách kombinuje nejčastěji s hořčíkem a s ocelí. Cílem je nabídnout maximální pevnost a co nejmenší hmotnost.
Specialitou jsou rámy z karbonového kompozitu, které se vyznačují svou lehkostí. Velkou nevýhodou je však vysoká cena a náchylnost na požkození, jelikož vrstvy karbonu se mohou po nárazu oddělit od sebe.

\subsubsection*{Převody a řazení}
Na kole tradičně najdeme dvě nebo tři ozubená kola vepředu a osm až deset pastorků vzadu. Obecně platí pravidlo, že čím méně vepředu a více vzadu, tím lépe. Polohu řetězu ovládá přehazovačka a přední přesmykač \textit{(nebo prostě přední a zadní přehazovačka)}. Ty se ovládají páčkami na řídítkách. 

Současným trendem je přesouvání převodů na zadní kazetu, kdy vepředu zůstane jen jediné ozubené kolo. Tento systém je značně pohodlný a vhodný pro závodní nasazení. Toto řazení je o něco rychlejší a plynulejší.

\newpage
\subsubsection*{Brzdy}
Brzdy se dělí vesměs na dva druhy, špalíkové \textit{("véčkové")} a diskové. Většina nových kol se naštěstí prodává již s diskovými brzdami, které nabízí lepší brzdící výkon. Špalíkové brzdy rychle ztrácí účinnost pokud se namočí nebo zašpiní.\\
\begin{figure}[h!]
    \centering
    \begin{minipage}[b]{0.45\textwidth}
        \includegraphics[width=\textwidth]{kotouc}
        \caption{Brzdový kotouč horského kola \newline{}\tiny{foto: Tektro}}
    \end{minipage}
    \hfill
    \begin{minipage}[b]{0.5\textwidth}
        \includegraphics[width=\textwidth]{desticky}
        \caption{Brzdové destičky s chladičem \newline{}\tiny{foto: Shimano}}
    \end{minipage}
\end{figure}

\noindent{}Specialitou jsou diskové brzdy pro sjezdová kola, která mají na kotoučích a brzdových destičkách namontované chladiče kvůli extrémním teplotám, které mohou při brzdění vzniknout.

\subsubsection*{Vidlice a tlumení}
Téměř všechna horská kola se prodávají s odpruženou přední vidlicí. Celoodpružená horská kola do těžšího terénu mají navíc tlumič zabudovaný v zadní části rámu.

Existují vesměs dva typy tlumičů podle jejich vnitřní funkce – levnější \textbf{pružinové}, které jsou určené spíše pro nenáročné jezdce kvůli jejich nízkému zdvihu a pomalým reakcím na terén; a \textbf{vzduchové}, které se dají nastavovat přesně podle individuálních potřeb a nabízí zpravidla vyšší zdvih, menší hmotnost a pohodlnější jízdu. Pokud s~kolem plánujete vyrazit do terénu, určitě zvažte investici do vzduchových tlumičů.

\newpage\section{Technika jízdy}
\subsection{Rovnováha a práce s těžištěm}
Zní to možná zvláštně, ale ten kdo umí kolo ovládat v pomalé rychlosti, umí jet i mnohem rychleji. Základem je udržet svoje těžíště přesně nad místem kudy chceme jet. Váš trup by měl zůstat na stejném místě, lokty a kolena mírně do široka, ať se kolo pod vámi může volně pohybovat. 

Rovněž důležité je to, kam se koukáte. Váš pohled by měl směřovat tam, kam chcete jet, cca. 1-4 metry před sebe. V žádném případě na přední kolo! 
Snažte se držet pedály v rovině a nedívat se na překážku, které se bojíme – např. kámen, do kterého nechceme narazit. Kam se díváte, tam jedete.

Na trénink rovnováhy se hodí třeba čára na parkovišti, prkno nebo obrubník, ideálně mírně do kopce. Zkuste se rozjet a postupně zpomalovat a přitom nesjet z čáry. Až budete mít pocit, že ztrácíte kontrolu, zkuste se šlápnutím do pedálů po čáře zase rozjet. Jedte ve stoje a vyvažujte kolem, ne trupem.
\subsection[Zatáčení]{Průjezd zatáčkou}
Nejčastější chybou je přijet rychle do zatáčky, až v ní začít brzdit a na poslední chvíli zatáčet a řešit, kam cesta pokračuje. 

Průjezd zatáčkou se dá rozdělit na 3 části – \textbf{příprava, průjezd a výjezd.}\cite{dressler}

\paragraph*{Příprava}hraje zásadní roli. Nejdříve je důležité zjistit jak moc a kam se bude váš směr měnit a jaký je povrch zatáčky. Pokud vám cokoli brání získat tyto informace, je potřeba předvídat a najíždět do zatáčky opatrně. 

Teď je důležité včas přibrzdit, a to ještě před nájezdem do zatáčky tak, abyste do ní přijeli už připravení. Zvolte si co nejplynulejší trasu, kterou pojedete. \textbf{Dívejte se tam, kam chcete jet.} Při nájezdu pootočte hlavou směrem k vnitřní straně zatáčky. Tím se vám podaří správně nastavit těžiště a kolo vás dovede vás ven ze zatáčky. 

\paragraph*{Průjezd}by měl být co nejplynulejší. Velmi důležité jsou pozice pedálů. V zatáčce se opřete do toho vnějšího a udržujte těžiště mírně vzadu. Při průjezdu si můžete plynule přibržďovat. Ale pozor, přední kolo se nesmí přestat točit! Pokud je zatáčka velmi ostrá, můžete si pomoct \textit{(řízeným!)} smykem zadního kola.
\paragraph*{Výjezd}je důležitý, hlavně pokud chcete zatáčkou projet co nejrychleji. Nejlepší je už v~druhé půlce zatáčky začit šlapat a~nasměrovat kolo kam chcete pokračovat. Přitom je důležité dát pozor na to, abyste si neškrtli vnitřním pedálem o terén a nespadli.

Určitě se vyplatí jezdit zatáčky ve stoje, umožňuje to lepší kontrolu nad kolem.

\newpage
\subsection{Brzdění}
Schopnost bezpečně zpomalit a zastavit je třetí nejzákladnější dovedností na kole, kterou by měl zvládat úplně každý. Při brzdění je dobré přenést těžiště trochu dozadu. Klíčové je rozložení brzdné síly mezi zadní \textit{a přední} kolo. \textbf{Správné je brzdit oběma brzdami současně.} Přední kolo umožňuje rychlejší zpomalení na rovné dráze, ale jeho smyk často končí pádem. Smyk tolik nevadí u zadního kola, kde stačí pustit zadní brzdu a kolo se odblokuje. 

\paragraph*{Rozložení brzdné síly}
Na rovném povrchu si můžeme dovolit rozložit brzdnou sílu skoro v poměru \textbf{1:1}. Přední vidlice se zmáčkne a posune vám těžiště o něco dopředu. To je potřeba upravovat posunutím trupu směrem k zadnímu kolu. 

Naopak v terénu je vhodné nechat přední kolo, aby se točilo a hledalo si cestu mezi nerovnostmi. Brzdnou sílu můžeme rozdělit přibližně jako 25\% na přední kolo a 75\% na zadní kolo. \newline

\noindent Na některých površích (třeba na štěrku nebo písku) je bezpečnější nechat kolo jet a zaměřit se na udržení kontroly nad kolem a těžištěm. 

\begin{figure}[h]
    \centering
    \includegraphics[width=0.8\textwidth]{drop}
    \caption{Obrázek pro dokreslení atmosféry \tiny{foto: vlastní}}
    \label{fig:drop}
\end{figure}
\newpage

\section{Závěr}
V této práci jste se dozvěděli základní historii, anatomii a druhy jízdních kol. V další části jsme se věnovali třem základním dovednostem v jízdě na horském kole. Doufám, že tato práce bude i přínosem pro její čtenáře, a že se třeba dozví něco nového. 

Na kole jezdím už velmi dlouho a rád se učím nové věci. Je to pro mě příjemné odreagování a příležitost být na čerstvém vzduchu. Témat okolo cyklistiky je mnohem víc, tohle byl spíše přehled toho nejzákladnějšího. 
\newline\noindent\newline\large{\textbf{Bavte se s kolem.}}
\vspace*{2cm}
\begin{figure}[h]
    \centering
    \includegraphics[width=0.6\textwidth]{dab}
\end{figure}
\newpage
\bibliographystyle{plain}
\begin{thebibliography}{10}
\bibitem{dressler}
DRESSLER, Josef, \textit{Škola kola.}, 2014

\bibitem{kolowiki}
Přispěvatelé Wikipedie, Jízdní kolo [online], Wikipedie: Otevřená encyklopedie, c2020, Datum poslední revize 4. 06. 2020, 08:05 UTC, [citováno 7. 06. 2020] \newline <\url{https://cs.wikipedia.org/w/index.php?title=J%C3%ADzdn%C3%AD_kolo&oldid=18591746}>

\bibitem{velikostiKol}
Bike-eshop.cz, Tabulka velikosti kola k postavě [online], [citováno 7. 06. 2020]\newline <\url{https://www.bike-eshop.cz/tabulka-velikosti-kola-k-postave}>
\end{thebibliography}
\vspace*{2cm}
\begin{figure}[h]
    \centering
    \includegraphics[width=0.8\textwidth]{kola}
\end{figure}

\newpage
\pagenumbering{gobble}
\vspace*{\fill}
\begin{figure}[h]
    \centering
    \includegraphics[width=0.2\textwidth]{ikona}
\end{figure}

\end{document}